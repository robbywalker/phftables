\documentclass[10pt]{article}

\usepackage{fullpage}
\usepackage{amsmath}
\usepackage{amssymb}

\newcommand{\bigo}{\mathrm{O}}
\newcommand{\PHF}{\mbox{$\mathsf{PHF}$}}
\newcommand{\PHFN}{\mbox{$\mathsf{PHFN}$}}

\renewcommand{\tabcolsep}{4pt}

\newtheorem{theorem}{Theorem}[section]
\newtheorem{lemma}{Lemma}[section]

\def\endmark{\hskip 2em$\blacksquare$\par}
\def\proof{\trivlist \item[\hskip \labelsep{\em Proof.\ }]}
\def\endproof{\null\hfill\endmark\endtrivlist}

\newcommand{\A}{\mbox{$\mathsf{A}$}}

% document start
\begin{document}

% title info
\title{Symbol Increase for Strength 3 Perfect Hash Families}

\author{Robert A. Walker II\\
Computer Science and Engineering\\
Arizona State University\\
P.O. Box 878809,\\
Tempe, AZ 85287, U.S.A.\\
{\tt robby.walker@gmail.com}
}
\date{}

\maketitle

%\begin{abstract}

\section{Theorems}

Will will consider a $\PHF(N;k,v,t)$ as an $N \times k$ array on $v$ symbols.
Then, to be a $\PHF$ the array must have the property that given an $N \times t$
sub-array, there is at least one row containing distinct symbols.

\begin{lemma}
If we are given a $\PHF(N;k,v,t)$, we may create a $\PHF(N;N,v+1,t)$ where every row contains
exactly one $x$ and every column contains exactly one $x$, where $x$ is the new symbol.
\end{lemma}

\begin{proof}
Construct the array using the first $N$ columns of the original array.  Replace the entry in the $i$-th row of the
$i$-th column with $x$.  This satisfies the second condition.  We need now verify that the new array represents a
perfect hash family.

Consider any $N \times t$ sub-array of the new array.  Each row in this sub-array contains exactly one $x$.  Consider
one of the rows (there is at least one) that contained distinct symbols in the original array.  Then, this row still
contains distinct symbols, since one distinct symbol has been replaced with an $x$, which is also distinct.
\end{proof}

Using this proof, we can now show the theorem.

\begin{theorem}
Given a $\PHF(N;k,v,3)$ we can create a $\PHF(N;k+N,v+1,3)$.
\end{theorem}
\begin{proof}
Construct a $\PHF(N;N,v+1,3)$ as in the Lemma.  Call this new array $\A'$.  Now, place that array adjacent to the original array (call it $\A$)
to form an $N \times k+N$ array.

Now we must consider any $N \by 3$ sub-array.  There are several cases to consider.

\begin{itemize}
\item 3 columns from $\A$: by definition, there will be at least one row of distinct symbols because $\A$ is a $\PHF$.

\item 3 columns from $\A'$: by the lemma, $\A'$ is a $\PHF$.

\item 2 columns from $\A$, one from $\A'$: DOES NOT WORK!

\item 1 column from $\A$, two from $\A'$: Subcase, the column from $\A$ has a copy in $\A'$

\end{proof}


\end{document}
